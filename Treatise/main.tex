% !TeX encoding = UTF-8
% !TeX spellcheck = en_GB
\documentclass[12pt,a4paper,draft]{report}
\usepackage[utf8]{inputenc}
\usepackage[T1]{fontenc}
\usepackage[british]{babel}
\usepackage{amsmath}
\usepackage{amsfonts}
\usepackage{amssymb}
\usepackage{graphicx}
\usepackage{fancyhdr}
\usepackage{setspace}
\usepackage[hidelinks]{hyperref}
\usepackage{float}
\usepackage[margin=3cm]{caption}
\usepackage{wrapfig}
\usepackage{listings}
\usepackage{booktabs}
%\usepackage[backend=biber, style=alphabetic, citestyle=authoryear]{biblatex}

\newfloat{kod}{h}{txt}
\floatname{kod}{K\^od}

\lstset{
	numbers=left,
	basicstyle=\footnotesize,
	captionpos=b,
	language=Python
}
% \renewcommand{\lstlistingname}{K\^od}% Listing -> Algorithm
% \renewcommand{\lstlistlistingname}{List of \MakeLowercase{\lstlistingname s}}% List of Listings -> List of Algorithms

\usepackage[left=3cm,right=3cm,top=2cm,bottom=3cm]{geometry}
\author{Bogdan Okreša Đurić}
\title{\naslov}
\pagestyle{fancy}

\newcommand{\institute}{SVEUČILIŠTE U ZAGREBU\\	FAKULTET ORGANIZACIJE I INFORMATIKE\\ V A R A Ž D I N}
\newcommand{\naslov}{Ususret ontologiji za mrežne računalne igre s ulogama namijenjene većem broju igrača: kritički osvrt na dosadašnji rad}
\newcommand{\autor}{Bogdan Okreša Đurić}
\newcommand{\predmet}{Odabrana poglavlja upravljanja znanjem}
\newcommand{\currentDate}{11. srpnja \the\year.}


\begin{document}
	
	
\begin{titlepage}
	\centering
		\textbf{\institute}\\[4cm]

	\raggedright
	{\large \autor}\\[6cm]
	
	\centering
		\textbf{\huge \textsc{\naslov}}\\[0.5cm]
		\textbf{\textsc{\predmet}}\\\vfill%[9cm]
		
		Varaždin, \the\year.
		
	\clearpage
	\thispagestyle{empty}
		
	\centering
		\textbf{\institute}\\[4cm]
		
	\raggedright
	{\large \autor}\\
	Broj indeksa: D-211/2014 \\
	Informacijske znanosti\\
	Sveučilišni doktorski studij\\[4.5cm]
	
	\centering
		\textbf{\huge  \textsc{\naslov}}\\[0.5cm]
		\textbf{DOKTORSKI RAD\\\textsc{\predmet}}\\[2cm]

	\raggedright
		\hspace{0.5\linewidth}\textbf{Mentor:}\\
		\hspace{0.5\linewidth}izv.prof.dr.sc. Markus Schatten\\\vfill%[5.5cm]
		
	\centering
		Varaždin, \currentDate
	
\end{titlepage}

% \begin{quote}
%     Devotion
% \end{quote}
% \pagebreak

\pagenumbering{roman}
% \chead[E]{{\predmet} - Doktorski rad}
% \lhead{}
% \rhead{}
% \thispagestyle{fancy}
\onehalfspacing

\tableofcontents

%\pagebreak

\clearpage
\listoffigures

\clearpage
\listoftables

\clearpage
\lstlistoflistings
%\listof{kod}{Isječci k\^oda}

\pagebreak

\pagenumbering{arabic}

%%% NEW CHAPTER - INTRODUCTORY NOTES %%%
\chapter{Introductory Notes}

\section{Motivation}

\section{Introduction}

\section{Conceptual Definitions}

\begin{table}[h]
    \centering
    \caption{Conceptual definitions of the most important concepts of this thesis}
    \label{tab:IntroConcepts}
    \begin{tabular}{p{4.5cm}|p{9cm}}
        \toprule
        \textbf{Concept Symbol} & \textbf{Concept Intension} \\\midrule
        Agent & \\
        Multiagent System & \\
        Large-Scale Multiagent System & \\
        Model & \\
        Metamodel & \\
        Ontology & \\
        \bottomrule
    \end{tabular}
\end{table}

\section{Literature Review of Related Research}

%%% NEW CHAPTER - SCIENTIFIC CONTRIBUTION %%%
\clearpage
\chapter{Scientific Contribution}



\section{Conceptual and Semantic Modelling}


\subsection{Related Research}


\subsection{Ontology Concepts}

\subsubsection{Classes}

\subsubsection{Object Properties}

\subsubsection{Data Properties}

\subsubsection{Inference Process}


\subsection{Ontology Example}



\section{Metamodelling}


\subsection{Related Research}


\subsection{Model Semantics}


\subsection{Model Syntax}


\subsection{Organisational Dynamics}


\subsection{Modelling Example}

%%% NEW CHAPTER - PRACTICAL CONTRIBUTION %%%
\clearpage
\chapter{Practical Contribution}

\section{Metamodelling Tool}

\section{Metamodel Implementation}

- elements used when implementing the metamodel

\subsection{Overcome Challenges}

\section{Generating Application Templates}

%%% NEW CHAPTER - FINAL REMARKS %%%
\clearpage
\chapter{Final Thoughts}



\section{Discussion}

\subsection{SWOT Analysis}

\begin{table}[h]
    \centering
    \begin{tabular}{p{0.45\textwidth}|p{0.45\textwidth}}
        \toprule
        \textsc{Strength} & \textsc{Weakness} \\
        text & text \\\midrule
        \textsc{Opportunity} & \textsc{Threat} \\
        text & text \\\bottomrule
    \end{tabular}
    \caption{SWOT analysis of the final practical contribution}
    \label{tab:my_label}
\end{table}


\section{Conclusions}



\bibliographystyle{ieeetr}
\bibliography{library}

\end{document}

% % % % % % % % % % % % % % % % % % % % % % % % % % % % % % % % %
% % % % % % % % % % % % % % % % % % % % % % % % % % % % % % % % %

%\begin{kod}
%	\centering
%	\texttt{<http://dragon.foi.hr/DragonOnto\#Char\_Luke>	<http://dragon.foi.hr/DragonOnto\#isMemberOf>	<http://dragon.foi.hr/DragonOnto\#Party\_Spam>}
%	\caption
%		[RDF trojka]
%		{RDF trojka koja opisuje odnos koncepata \texttt{Char\_Luke}, \texttt{isMemberOf} i \texttt{Party\_Spam}}
%	\label{kod:RDF_pocetni}
%\end{kod}

%\begin{figure}
%	\centering
%	\includegraphics[width=\linewidth]{RDF1}
%	\caption
%		[RDF graf]
%		{Graf koji predstavlja RDF trojku iz k\^oda \ref{kod:RDF_pocetni}}
%	\label{fig:RDF_pocetni}
%\end{figure}

%\begin{table}
%	\caption{Neke od klasa koje su korištene za modeliranje svijeta igre The Mana World}
%	\label{tbl:Primjer_klase}
%	\begin{center}
%		\begin{tabular}{|p{0.20\linewidth}|p{0.20\linewidth}|p{0.5\linewidth}|}
%			\hline \textbf{Klasa} & \textbf{Nadklasa} & \textbf{Opis} \\ 
%			
%			\hline \texttt{CharClass} & \texttt{owl:Thing} & Sadrži sve moguće klase likova igrača, koje su određene prema rasporedu bodova vještina pri stvaranju lika igrača.
%			\hline 
%		\end{tabular} 
%	\end{center}
%\end{table}
